Since the Higgs boson was first observed in 2012 by the CMS and ATLAS collaborations~\cite{Aad:2012tfa, Chatrchyan:2012xdj, Chatrchyan:2013lba}, characterizing its properties has remained one of the highest priorities of the LHC research program. 
The Standard Model predicts values for many properties of the Higgs boson, including the strength of its coupling to the other elementary particles.
Physics beyond the Standard Model, such as mechanisms of mass generation other than spontaneous symmetry breaking FIXME:CITE, could modify these coupling strengths. 
Consequently, precise measurements of the Higgs boson's coupling to elementary particles are of great interest: any deviation from the Standard Model prediction could be indicative of the presence of new physics.
%
\subsection{The Top Quark Yukawa Coupling}
The coupling of the Higgs boson to the top quark, called the top quark Yukawa coupling, is of particular interest from a theoretical standpoint~\cite{why_care_top_yukawa}.
FIXME: list reasons
A primary means of constraining the top quark Yukawa coupling is through

\subsection{\ttH~Production as a Probe of the Top Quark Yukawa Coupling}
