%
%
% UCSD Doctoral Dissertation Template
% -----------------------------------
% http://ucsd-thesis.googlecode.com
%
%


%% REQUIRED FIELDS -- Replace with the values appropriate to you

% No symbols, formulas, superscripts, or Greek letters are allowed
% in your title.
\title{Observation of Higgs boson production in association with a top quark-antiquark pair in the diphoton decay channel}

\author{Samuel James May}
\degreeyear{\the\year}

% Master's Degree theses will NOT be formatted properly with this file.
\degreetitle{Doctor of Philosophy}

\field{Physics}
%\specialization{Anthropogeny}  % If you have a specialization, add it here

%\chair{Professor Frank W{\"u}rthwein}
% Uncomment the next line iff you have a Co-Chair
%\cochair{Professor Cochair Semimaster}
%
% Or, uncomment the next line iff you have two equal Co-Chairs.
%\cochairs{Professor Frank W{\"u}rthwein}{Professor Avi Yagil}
\chair{Professor Avraham Yagil}

%  The rest of the committee members  must be alphabetized by last name.
\othermembers{
Professor Garrison Cottrell\\
Professor Aneesh Manohar\\
Professor Julian McAuley\\
Professor Frank W{\"u}rthwein 
}
\numberofmembers{5} % |chair| + |cochair| + |othermembers|


%% START THE FRONTMATTER
%
\begin{frontmatter}

%% TITLE PAGES
%
%  This command generates the title, copyright, and signature pages.
%
\makefrontmatter

%% DEDICATION
%
%  You have three choices here:
%    1. Use the ``dedication'' environment.
%       Put in the text you want, and everything will be formated for
%       you. You'll get a perfectly respectable dedication page.
%
%
%    2. Use the ``mydedication'' environment.  If you don't like the
%       formatting of option 1, use this environment and format things
%       however you wish.
%
%    3. If you don't want a dedication, it's not required.
%
%
\begin{dedication}
 To my mom and dad, who have always encouraged my curiosity and supported my education. 
\end{dedication}


% \begin{mydedication} % You are responsible for formatting here.
%   \vspace{1in}
%   \begin{flushleft}
% 	To me.
%   \end{flushleft}
%
%   \vspace{2in}
%   \begin{center}
% 	And you.
%   \end{center}
%
%   \vspace{2in}
%   \begin{flushright}
% 	Which equals us.
%   \end{flushright}
% \end{mydedication}



%% EPIGRAPH
%
%  The same choices that applied to the dedication apply here.
%
\begin{epigraph} % The style file will position the text for you.
  \emph{Sometimes science is more art than science.}\\
  ---Rick Sanchez
\end{epigraph}

% \begin{myepigraph} % You position the text yourself.
%   \vfil
%   \begin{center}
%     {\bf Think! It ain't illegal yet.}
%
% 	\emph{---George Clinton}
%   \end{center}
% \end{myepigraph}


%% SETUP THE TABLE OF CONTENTS
%
\tableofcontents
\listoffigures  % Comment if you don't have any figures
\listoftables   % Comment if you don't have any tables



%% ACKNOWLEDGEMENTS
%
%  While technically optional, you probably have someone to thank.
%  Also, a paragraph acknowledging all coauthors and publishers (if
%  you have any) is required in the acknowledgements page and as the
%  last paragraph of text at the end of each respective chapter. See
%  the OGS Formatting Manual for more information.
%
\begin{acknowledgements}
    Many people, more than I can list here, have helped me in completing my PhD.
    I am grateful to everyone who has helped me along the way -- my work would not have been possible without you.

    My mom and dad deserve to be thanked first. They encouraged my education throughout my entire life and made me feel that my interests were worth pursuing.
    My cousin, Christopher Betancourt, also deserves to be thanked for getting me interested in particle physics in the first place.

    Second, I would like to thank my undergraduate research advisors at UCLA: Jon Aurnou and Bob Cousins.
    Jon was my first research advisor and taught me how to think like a researcher.
    Bob gave me my introduction to particle physics, both in the classroom, where he taught the first course on particle physics I took, and in a research environment, where he mentored me on a project studying the statistical method of unfolding in the context of high energy physics (and taught me a great deal of statistics along the way).
    
    Third, I would like to thank my thesis advisors, Frank Wuerthwein and Avi Yagil.
    Frank encouraged me to explore a wide range of projects, including ones not necessarily related to physics.
    Avi consistently challenged me to be a better physicist, and I learned more because of it.
    Claudio Campagnari also deserves to be thanked along with Frank and Avi, as he acted as my unofficial third advisor.
    Frank, Avi, and Claudio each helped me develop my mind as a physicist through a variety of projects in which they mentored me.

    Fourth, I would like to thank ``Surf and Turf'' (SNT) group from UCSD, UCSB, Fermilab, Boston University, and the University of Nebraska.
    Everyone in SNT during my time at UCSD has helped me in some way, even if just indirectly.
    A few people deserve special thanks. 
    Vince Welke, Dan Klein, Bobak Hashemi, and Mark Derdzinski helped me get started on my first project in grad school.
    Slava Krutelyov never failed to answer my questions about the fine details of CMSSW software.
    Philip Chang mentored me on my first foray into machine learning (and was always willing to discuss interesting papers \& ambitious new ideas).
    I would also like to especially thank Nick Amin: not only was Nick the de facto expert on every SNT repository, he always went above and beyond when answering my many questions. 

    Last, I would like to thank those who specifically mentored me on my thesis project.
    Dominick Olivito and Giovanni Zevi Della Porta taught me how a physics analysis was done in CMS. They always made time to answer my questions, often taking time out of their evenings to do so.
    Bennett Marsh helped in the initial stages of this project, and helped me in understanding the code used by the CMS Higgs to Gamma Gamma group.
    Frank Golf provided valuable insights and interesting discussions in the initial stages of this project.
    Avi and Claudio mentored me on this project over the course of two and a half years.
    The entire CMS Higgs to Gamma Gamma group must be thanked as well, as they not only provided feedback on my work, but contributed many of the common tools shared by all Higgs to Gamma Gamma analyses in CMS, without which this analysis would not have been possible.
    In particular, Shervin Nourbaksh, Seth Zenz, Julie Malcles, and Ed Scott, who acted as conveners of the Higgs to Gamma Gamma group during the time I worked on this thesis, provided valuable insight on my work.
    Meng Xiao, Andrei Gritsan, and Mehmet Ozgur Sahin developed the CP measurement component of the analysis and provided feedback on the analysis as a whole.
    Hualin Mei deserves to be thanked in particular: Hualin served as both a mentor and an (exceptionally reliable and hard-working) teammate to me in developing this analysis, often staying up late to answer my questions or discuss the finer details of the analysis.

    %Chapter 1 introduces the basics of particle physics and motivates the measurement of the \ttH process, in my own words. 
    %
    %Chapter 2 explains quantum field theory and the Standard Model of particle physics, in my own words.
    %
    %Chapter 3 explains the physics of proton-proton collisions, in my own words.
    %
    Chapter 4 describes the Large Hadron Collider and Compact Muon Solenoid detector.
    The figures shown in Chapter 4 are taken from the following results: ``Performance and track-based alignment of the Phase-1 upgraded CMS pixel detector'', \emph{CMS-CR-2017-256} (2017), ``CMS Luminosity -- Public Results'', \emph{https://twiki.cern.ch/twiki/bin/view/CMSPublic/LumiPublicResults} (2020), ``The CMS Experiment at the CERN LHC'', \emph{Journal of Instrumentation} (2008), ``Description and performance of track and primary-vertex reconstruction with the CMS tracker'', \emph{Journal of Instrumentation} (2014), and ``The CMS trigger system'', \emph{Journal of Instrumentation} (2017), and were produced by other members of the CMS Collaboration.
    
    Chapter 5 describes the event reconstruction pipeline in CMS with a particular focus on photon reconstruction.
    The figures shown in Chapter 5 are taken from the following publications: ``Performance of Photon Reconstruction and Identification with the CMS Detector in Proton-Proton Collisions at sqrt(s) = 8 TeV'', \emph{Journal of Instrumentation} (2015), ``Jet energy scale and resolution in the CMS experiment in pp collisions at 8 TeV'', \emph{Journal of Instrumentation} (2017), ``Identification of heavy-flavour jets with the CMS detector in pp collisions at 13 TeV'', \emph{Journal of Instrumentation} (2018), and ``Measurements of Higgs boson properties in the diphoton decay channel at $\sqrt{s} = 13$ TeV'', \emph{CMS-PAS-HIG-19-015} (2020), and were produced by other members of the CMS Collaboration, particularly those involved in the CMS Higgs to Gamma Gamma working group.

    Chapter 6 describes the \ttH analysis documented in ``Measurements of $\mathrm{t\bar{t}}$H production and the CP structure of the Yukawa interaction between the Higgs boson and top quark in the diphoton decay channel'' \emph{Phys. Rev. Lett.} 125 (2020), with a focus on the aspects to which I contributed most directly, but relies heavily from the work of other members of the CMS Collaboration and the CMS Higgs to Gamma Gamma working group, without whom this analysis would not have been possible.
    My primary individual contributions to this work included the following: implementation of the data-driven description of multi-jet and $\gamma$ + jet backgrounds, studies of agreement between data and simulation, training and optimization of the deep neural networks and boosted decision trees used for signal region definition, and implementation of some systematic uncertainties. 
    Figures~\ref{fig:tth_sig_model},\ref{fig:tth_bkg_functions}, and~\ref{fig:tth_bkg_models} show the results of the signal and background models for the \ttH analysis, and were produced by Hualin Mei.
    Figure~\ref{fig:tth_impacts} shows the impact of systematic uncertainties on the measurement of $\mu_{\ttH}$, and was produced by Hualin Mei.
    Figures~\ref{fig:tth_obs_sr_weighted} and~\ref{fig:tth_llr} show the observed results of the \ttH analysis, and were also produced by Hualin Mei.
    Figure~\ref{fig:tth_cp} shows the result of the \ttH CP measurement and was produced by Meng Xiao.
    %
    %Chapter 7 summarizes the results of this dissertation and provides my own speculation on considerations for future research in particle physics.
\end{acknowledgements}


%% VITA
%
%  A brief vita is required in a doctoral thesis. See the OGS
%  Formatting Manual for more information.
%
\begin{vitapage}
\begin{vita}
  \item[2016] B.~S. in Physics, University of California, Los Angeles
  \item[2016-2017] Graduate Teaching Assistant, University of California San Diego
  \item[2018] M.~S. in Physics, University of California San Diego
  \item[2020] Ph.~D. in Physics, University of California San Diego
  %\item[2002] B.~S. in Mathematics \emph{cum laude}, University of Southern North Dakota, Hoople
  %\item[2002-2007] Graduate Teaching Assistant, University of California, San Diego
  %\item[2007] Ph.~D. in Mathematics, University of California, San Diego
\end{vita}
\begin{publications}
  \item CMS Collaboration, ``Measurements of Higgs boson properties in the diphoton decay channel at $\sqrt{s}$ = 13 TeV'', \emph{CMS-PAS-HIG-19-015}, 2020.
  \item CMS Collaboration, ``Measurements of $\mathrm{t\bar{t}}$H production and the CP structure of the Yukawa interaction between the Higgs boson and top quark in the diphoton decay channel'', \emph{Phys.Rev.Lett.} 125 (2020) 6, 061801.
  \item Robert D. Cousins, \textbf{Samuel May}, and Yipeng Sun, ``Should unfolded histograms be used to test hypotheses?'', \emph{arXiv:1607.07038}, 2016.
  %\item Your Name, ``A Simple Proof Of The Riemann Hypothesis'', \emph{Annals of Math}, 314, 2007.
  %\item Your Name, Euclid, ``There Are Lots Of Prime Numbers'', \emph{Journal of Primes}, 1, 300 B.C.
\end{publications}
\end{vitapage}


%% ABSTRACT
%
%  Doctoral dissertation abstracts should not exceed 350 words.
%   The abstract may continue to a second page if necessary.
%
\begin{abstract}
  This dissertation presents the first observation of Higgs boson production in association with a top quark-antiquark pair in the diphoton decay channel, with a significance of 6.6 standard deviations.
  The measurement is performed with a dataset of 13 TeV proton-proton collisions recorded by the Compact Muon Solenoid (CMS) detector at the CERN Large Hadron Collider (LHC), corresponding to an integrated luminosity of 137 \fbinv.

\end{abstract}


\end{frontmatter}
