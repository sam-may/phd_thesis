\chapter{Conclusion} \label{chap:conclusion}

Measurements of the production cross section and signal strength of Higgs boson production in association with a top quark-antiquark pair in the diphoton decay channel were presented.
With an observed significance of 6.6 standard deviations, this is the first observation of \ttH in a single decay channel of the Higgs boson.
The observed cross section times branching fraction of $1.56^{+0.34}_{-0.32}$~fb is compatible with the SM prediction of $1.13^{+0.08}_{-0.11}$~fb, and the observed signal strength of $1.38^{+0.36}_{-0.29}$ is compatible with the SM prediction of unity.

Although the measurements of \ttH production are so far compatible with the SM predictions, more precise measurements are necessary to determine whether the interactions of the Higgs boson are truly compatible with those predicted by the SM.
Many BSM theories predict deviations from the SM couplings at a percent level~\cite{Dawson:2013bba}, while the precision of the measurement presented in this thesis is around an order of magnitude higher.
As the uncertainty of this measurement is still heavily statistically-dominated, simply repeating the analysis with the larger datasets expected from Run 3 of the LHC and the HL-LHC will improve our ability to judge whether the Higgs couplings are indeed SM-like.
The \Hgg decay channel will especially benefit from increased luminosity, due to its low systematic uncertainties relative to other decay modes, like the decay of the Higgs boson to bottom quarks.
Beyond increasing the integrated luminosity of the datasets, the sensitivity of this measurement can be improved in multiple ways, including:
through further study of advanced machine learning algorithms used to identify signal-like events, such as those described in Sec.~\ref{sec:tth_dnns},
through the continued development of creative methods for improving the description of the SM background processes, like that of Sec.~\ref{sec:tth_datadriven},
and through the use of methods to decrease the experimental systematic uncertainties associated with the measurement, such as the chained quantile regression method utilized to improve the agreement between simulation and data, as described in Sec.~\ref{sec:evt_photon_ss}.

The strategies developed in this analysis are broadly applicable to measurements other than just that of \ttH, especially measurements involving \Hgg in the final state.
In particular, a similar strategy may be adopted for searches for new physics with \Hgg in the final state, such as a search for the Higgs boson acting as a flavor-changing neutral current~\cite{agashe2013snowmass} in decays of the top quark to a Higgs boson and a light-flavor quark.
The strategies may be similarly adopted to measurements of other SM Higgs production modes, such as that of double-Higgs production, which may be a sensitive probe to the Higgs self-coupling~\cite{DiVita:2017eyz}.
In the same spirit that new physics may present itself in modified interactions of the top quark and the Higgs boson~\cite{why_care_top_yukawa}, it might also present itself in a modified Higgs self-coupling~\cite{Kanemura:2002vm}.
Precision measurements of the properties of the Higgs boson will continue to test the limits of the standard model's accuracy and provide a complementary approach to direct searches for new physics.

The results from Run 3 and the HL-LHC will provide unprecedented precision on the properties of the Higgs boson.
Either the results will continue to be compatible with the SM predictions, giving us further validation of one of the most successful theories in all of physics, or the results will show disagreement with the SM predictions, giving the field of particle physics a clear area to focus on in the goal of discovering new physics beyond the standard model.
%The author is willing to bet more money than he owns that if any hints of new physics come from the LHC, they will come in the form of a Higgs self-coupling
