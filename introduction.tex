\chapter{Introduction} \label{chap:introduction}

The standard model (SM) of particle physics is, to-date, the most successful theory in describing the known elementary particles of the universe and their interactions.
It describes three of the four known fundamental forces: electromagnetic, weak, and strong; however, does not provide a description of gravity.
Nearly every single experimental test of the standard model is in agreement with theory, with a few notable exceptions, described in greater detail in Sec.~\ref{sec:theory_sm}.
One of the key features of the SM is the Higgs mechanism~\cite{Higgs:1964pj,Englert:1964et,Guralnik:1964eu}, which explains how particles obtain mass.
An associated particle, the Higgs boson, is also predicted as a consequence of the Higgs mechanism.
The Higgs boson was discovered by the ATLAS and CMS collaborations in 2012~\cite{Aad:2012tfa, Chatrchyan:2012xdj, Chatrchyan:2013lba} with data collected during Run 1 of the Large Hadron Collider (LHC).
Since the discovery of the Higgs boson, characterizing its properties has remained one of the highest priorities of research in particle physics.

The results presented in this thesis describe the first observation of Higgs boson production in association with a top quark-antiquark pair (\ttH) in a single decay channel (in which the Higgs boson decays into a pair of photons)~\cite{tth_observation}.
The observation is performed with data collected during Run 2 of the LHC with the CMS detector.
Studying \ttH production allows us to understand the interaction between the Higgs boson and the top quark, of particular interest from a theoretical point of view as many theories of physics beyond the standard model (BSM) may present themselves in the form of modified (relative to the SM prediction) interactions between the Higgs boson and the top quark~\cite{why_care_top_yukawa}.
This thesis is organized as follows.

Chapter 2 provides an introduction to quantum field theory and the standard model of particle physics, with a focus on aspects related to the Higgs boson. It also describes the known shortcomings of the SM and how these shortcomings motivate measurements like that of \ttH. 

Chapter 3 provides an introduction to the physics of proton-proton collisions, necessary for studying the Higgs boson at the LHC, which collides bunches of protons.

Chapter 4 gives an overview of the LHC and the CMS detector, a multi-purpose apparatus designed to study a wide variety of particles and their underlying physics.

Chapter 5 describes how the raw data from the CMS detector is reconstructed into high-level physics objects suitable for analysis, with a focus on aspects relevant to \Hgg analyses.

Chapter 6 describes the \ttH analysis documented in~\cite{tth_observation}, with a focus on the aspects to which I contributed most directly.

Chapter 7 draws conclusions, provides perspective on how these results more broadly fit into the field particle physics, and speculates on future work which may build upon these results. 

%Nearly every single experimental test of the standard model is in agreement with theory, though a notable exception is the observation of neutrino oscillations~\cite{Fukuda:1998mi} which implies that they are massive -- in contradiction with the prediction from the SM that neutrinos are massless.
%Other shortcomings of the SM include its  


