The Large Hadron Collider (LHC) is a hadron accelerator and collider located in a 27~km underground tunnel on the French-Swiss border, near Geneva.
Its primary physics goal is to reveal physics beyond the Standard Model~\cite{Evans_2008}.
The LHC hosts a variety of experimental collaborations, including the CMS experiment~\cite{Chatrchyan:2008aa}, the ATLAS experiment~\cite{Aad:2008zzm}, the LHCb experiment~\cite{Alves:2008zz}, and the ALICE experiment~\cite{Aamodt:2008zz}.
The physics goals of the CMS and ATLAS experiments are identical, namely to search for the presence of new physics beyond the Standard Model and to make precision measurements of the properties of the Higgs boson.
The LHCb experiment focuses on studies of CP violation and rare decays of b hadrons, while the ALICE experiment focuses on studying the strong interactions of QCD.
The LHC collides both proton and lead (Pb) ion beams.
The CMS, ATLAS, and LHCb experiments are designed to study physics from proton-proton collisions, while the ALICE experiment utilizes the Pb-Pb collision data.
This thesis focuses on results obtained in proton-proton collisions with the CMS detector.

Given that the primary research goal of the LHC is to disover physics beyond the Standard Model, protons are the natural choice for the beam content, rather than electron-positron beams, as used in other colliders like the Large Electron Positron (LEP) collider~\cite{Taylor:2017edx}.
Two primary advantages provided by proton beams as a disovery tool are the fact that protons are composite particles, rather than elementary particles like electrons, and the fact that proton beams can more easily be sustained at higher energies.

Protons are primarly composed of up quarks, down quarks, and gluons, with the subcomponents collectively referred to as partons.
An interaction between two protons is then more precisely an interaction between two partons from each proton.
The partons each carry some fraction of the proton's energy, effectively providing a range of collision energies.
Since the energy scale of new physics is not precisely known, it is more desirable to have collisions occuring at a range of energies, as naturally occurs in proton collisions.
The proton beams at the LHC are accelerated to energies of 6.5 TeV, providing a center of mass energy of 13 TeV.
New physics which might exist at the TeV scale is then accessible, in principle, through the range of energies provided by proton-proton collisions.
In contrast, in order to achieve sensitivity to possible new physics at a range of energies with electron-positron beams, one would need to manually change the collision energies.
Broadly speaking, proton-proton colliders are better suited for discovery of particles whose exact mass is not known, while electron-positron colliders are better suited for precision measurements of particles whose mass is precisely known.

Proton beams are more easily sustained at high energies than electron-positron beams due to the fact that protons dissipate less energy through synchrotron radiation.
The energy emitted per unit time per unit solid angle for an accelerated charged particle is inversely proportional to the particle's mass~\cite{Jackson:100964}:
\begin{equation}
    P \propto \frac{1}{m^2}
    \label{eqn:bremm}
    %\caption{Mass dependence of the energy radiated per unit time for an accelerated charged particle.}
\end{equation}
Given that the mass of an electron is on the order of 1 MeV and the mass of a proton is on the order of 1 GeV, the mass dependence of Eqn~\ref{eqn:bremm} contributes a factor of $\mathcal O(10^6)$ difference in the energy lost to synchrotron radiation between protons and electrons accelerated in a circular trajectory.
The challenges associated with energy lost to synchrotron radiation in electron beams are reflected in the difference of center-of-mass energies achieved by the LHC (13 TeV) and LEP (209 GeV).

The LHC accelerates protons in groups known as bunches. The probability for any two protons to interact is extremely low; this is counteracted by increasing the number of protons contained in each bunch, $N_b$.
$N_b$ cannot be increased without bound and is limited by nonlinear beam-beam interactions which occur when two bunches collide with each other.
At the LHC, the maximum attainable bunch size is $N_b = 1.15 \times 10^{11}$.

Although the large circumference (27 km) of the LHC results in longer and more expensive beam pipes, it provides several advantages for sustaining high energy collisions.
Proton beams must be constrained to travel in a circular shape through the presence of a large magnetic field.
The beams at the LHC require a magnetic field of over 8 T to stay on track; however, this requirement would be significantly higher with a smaller circumference collider -- a larger circumference reduces the curvature and allows for higher energy beams at a given magnetic field strength.
Sustaining such a large magnetic field also presents challenges: an extremely large current (over $10^4$ A) is required to produce the magnetic field.
For such a high current, superconducting magnets are required: niobium-titanium superconducting electromagnets are used in the LHC, and superconductors stable at higher magnetic fields are extremely expensive. 

Even with billions of protons per bunch, only $\mathcal O(10-100)$ pairs of protons will actually collide with each other in a given bunch crossing. % FIXME: update with reference to pileup histogram
Fig~\ref{fig:cms_pileup} shows the distribution of the number of proton-proton interactions, also referred to as the ``pileup'', in the CMS detector during Run 2 of the LHC. 
The bunches are spaced by a distance corresponding to a time of 25 ns between bunches.
In general, the more closely bunches are grouped together, the more collisions can be recorded; however, the bunch spacing is limited by considerations like experiments' temporal resolution -- bunches must be sufficiently separated to allow each experiment to distinguish between consecutive collisions.

\begin{figure} [htbp!]
    \centering
    \includegraphics[width=0.6\linewidth]{figures/cms/pileup_allYears_run2.pdf}
    \caption{Mean number of interactions per bunch crossing recorded by the CMS detector during Run 2 of the LHC. Taken from~\cite{public_lumi}.}
    \label{fig:cms_pileup}
\end{figure}

During Run 2 of the LHC, the CMS detector recorded 150~\fbinv of data from proton-proton collisions.
A subset of that data is verified to have stable detector performance and marked as usable for physics analysis, amounting to 137~\fbinv of data used in the analysis discussed in this work.

\begin{figure} [htbp!]
    \centering
    \includegraphics[width=\linewidth]{figures/cms/int_lumi_allcumulative_pp_run2_cutout.pdf}
    \caption{Total luminosity delivered by the LHC (blue) and total luminosity recorded by the CMS detector (yellow) during Run 2 of the LHC. Taken from~\cite{public_lumi}.}
    \label{fig:cms_lumi}
\end{figure}
