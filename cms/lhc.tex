The Large Hadron Collider (LHC) is a hadron accelerator and collider located in a 27km underground tunnel on the French-Swiss border, near Geneva.
Its primary physics goal is to reveal physics beyond the Standard Model~\cite{Evans_2008}.
The LHC hosts a variety of experimental collaborations, including the CMS experiment~\cite{Chatrchyan:2008aa}, the ATLAS experiment~\cite{Aad:2008zzm}, the LHCb experiment~\cite{Alves:2008zz}, and the ALICE experiment~\cite{Aamodt:2008zz}.
The physics goals of the CMS and ATLAS experiments are identical, namely to search for the presence of new physics beyond the Standard Model and to make precision measurements of the properties of the Higgs boson.
The LHCb experiment focuses on studies of CP violation and rare decays of b hadrons, while the ALICE experiment focuses on studying the strong interactions of QCD.
The LHC collides both proton and lead (Pb) ion beams.
The CMS, ATLAS, and LHCb experiments are designed to study physics from proton-proton collisions, while the ALICE experiment utilizes the Pb-Pb collision data.
This thesis focuses on results obtained in proton-proton collisions with the CMS detector.

Given that the primary research goal of the LHC is to disover physics beyond the Standard Model, protons are the natural choice for the beam content, rather than electron-positron beams as used in other colliders like the Large Electron Positron (LEP) collider~\cite{Taylor:2017edx}.
The advantages provided by proton beams as a disovery tool include 
