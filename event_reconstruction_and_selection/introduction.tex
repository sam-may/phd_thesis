The raw data recorded by the CMS detector for a single bunch crossing is typically not yet suitable for high-level physics analysis.
Several layers of abstraction transform the data from the raw detector readout into high-level physics objects.
The first layer of abstraction uses the particle flow (PF) algorithm~\cite{particle_flow}, which combines information from each of the CMS subdetectors (the tracker, ECAL, HCAL, and muon system) in an attempt to reconstruct every particle in the event (PF candidates).
This first step is common to most physics analyses performed within the CMS experiment and is described in Section~\ref{sec:evt_pf}.
Physics objects are then refined further by placing quality requirements on the PF candidates.
This second step is often analysis-specific and the details are dictated by the individual needs of the given analysis.
The details of physics object definition specific to the \ttH analysis are described in Sections~\ref{sec:evt_photon}--\ref{sec:evt_met}.
