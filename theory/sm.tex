The standard model of particle physics is a quantum field theory which describes three of the four known fundamental forces and all known elementary particles.
The SM describes the electromagnetic, strong, and weak interactions, but does not provide a description of gravity.

In particular, the SM is a gauge field theory, meaning its Lagrangian is invariant under certain local transformations.
Gauge fields are discussed in greater detail in Sec.~\ref{sec:theory_gauge}.
In Sec.~\ref{sec:theory_qed}, we will see that imposing local gauge invariance on the Dirac Lagrangian gives rise to quantum electrodynamics.
Sec.~\ref{sec:theory_qcd} describes the strong interaction and Sec.~\ref{sec:theory_ewk} describes the electroweak interaction, the unification of the electromagnetic and weak interactions.
Finally, in Sec.~\ref{sec:theory_ssbhm} we will see how the Higgs mechanism allows for massive gauge fields and subsequently generates the masses of the gauge bosons and fermions.

\subsection{Gauge Fields} \label{sec:theory_gauge}
For an arbitrary Lagrangian made of a single field variable $\psi$, suppose we impose that its resulting field equations be invariant under the local phase transformation
\begin{equation} \label{eqn:local_phase}
    \psi \to e^{i q \theta(x)} \psi.
\end{equation}
This is deemed a ``local'' phase transformation as the phase $\theta$ is be a function of $x^\mu$.
In the case that $\theta$ is a constant, we deem this a ``global'' phase transformation.
A Lagrangian that is invariant under the transformation in Eqn.~\ref{eqn:local_phase} is said to be \emph{gauge invariant}.
More generally, theories which are invariant under gauge transformations are called \emph{gauge theories}.
As Sec.~\ref{sec:theory_qed} will show, quantum electrodynamics is an abelian gauge theory under the symmetry group U(1), with a single gauge field.
The SM as a whole is a non-abelian gauge theory under the symmetry group U(1) $\times$ SU(2) $\times$ SU(3), with a total of twelve gauge fields corresponding to the spin-1 bosons: the photon, the three massive weak bosons, and the eight gluons.

Gauge theories are particularly attractive from a theoretical standpoint for several reasons.
First, demanding gauge invariance seems reasonable a priori -- the transformation in Eqn.~\ref{eqn:local_phase} is simply a change in the coordinate system we use to define the field $\psi$, and the physics of the universe should be independent of the particular choice of coordinates we use to describe it.
Second, gauge theories have been proven to be renormalizable~\cite{tHooft:1971qjg}, also a reasonable requirement for a theory we hope will describe the universe.

\subsection{Quantum Electrodynamics} \label{sec:theory_qed}
Starting with the Dirac Lagrangian (Eqn.~\ref{eqn:dirac_lagrangian}), suppose we impose that its resulting field equations must be invariant under a local phase transformation (as given by Eqn.~\ref{eqn:local_phase}).
Initially, the Dirac Lagrangian is not invariant under the local phase transformation, as an extra term from the derivative of $\theta$ appears:
\begin{equation} \label{eqn:ld_var}
    \mathcal L \to \mathcal L - q (\partial_\mu \theta)\bar{\psi} \gamma^\mu \psi.
\end{equation}
The situation can be remedied with the introduction of a vector field $A_\mu$ which transforms under Eqn.~\ref{eqn:local_phase} as
\begin{equation} \label{eqn:local_phase_field}
    A_\mu \to A_\mu + \partial_\mu \theta.
\end{equation}
The resulting Lagrangian,
\begin{equation} \label{eqn:ld_inv}
    \mathcal L = \mathcal L_{\text{Dirac}} + q \bar{\psi} \gamma^\mu \psi A_\mu
\end{equation}
is gauge invariant as the second term in Eqn.~\ref{eqn:ld_inv} cancels exactly with the second term in Eqn.~\ref{eqn:ld_var}.

The vector field $A_\mu$ which has been added to the Lagrangian implies the existence of an associated spin-1 particle.
In principle, we must also include a free term for the field $A_\mu$: it is natural to start with the Proca Lagrangian (Eqn.~\ref{eqn:proca_lagrangian}) which describes the dynamics of free spin-1 particles.
It can be shown~\cite{Griffiths:2008zz} that the mass term in the Proca Lagrangian is not invariant under Eqn.~\ref{eqn:local_phase_field}: this can be interpreted as a requirement that this new vector field $A_\mu$ must be massless.
The full Lagrangian becomes
\begin{equation}
    \mathcal L_{\text{QED}} = \bar{\psi}(i \gamma_\mu \partial_\mu - m) \psi - \frac{1}{4} F^{\mu\nu}F_{\mu\nu} + q (\bar{\psi}\gamma^\mu \psi)A_\mu,
\end{equation}
which can be identified as the Lagrangian for quantum electrodynamics.

\subsection{Quantum Chromodynamics} \label{sec:theory_qcd}

\subsection{Electroweak Interactions} \label{sec:theory_ewk}

\subsection{Spontaneous Symmetry Breaking \& The Higgs Mechanism} \label{sec:theory_ssbhm} 
