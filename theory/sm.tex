The standard model of particle physics is a quantum field theory which describes three of the four known fundamental forces and all known elementary particles.
The SM describes the electromagnetic, strong, and weak interactions, but does not provide a description of gravity.

In particular, the SM is a gauge field theory, meaning its Lagrangian is invariant under certain local transformations.
Gauge fields are discussed in greater detail in Sec.~\ref{sec:theory_gauge}.
In Sec.~\ref{sec:theory_qed}, we will see that imposing local gauge invariance on the Dirac Lagrangian gives rise to quantum electrodynamics.
Sec.~\ref{sec:theory_qcd} describes the strong interaction and Sec.~\ref{sec:theory_ewk} describes the electroweak interaction, the unification of the electromagnetic and weak interactions.
Finally, in Sec.~\ref{sec:theory_ssbhm} we will see how the Higgs mechanism allows for massive gauge fields and subsequently generates the masses of the gauge bosons and fermions.

\subsection{Gauge Fields} \label{sec:theory_gauge}
For an arbitrary Lagrangian made of a single field variable $\psi$, suppose we impose that its resulting field equations be invariant under the local phase transformation
\begin{equation} \label{eqn:local_phase}
    \psi \to e^{i q \theta(x)} \psi.
\end{equation}
This is deemed a ``local'' phase transformation as the phase $\theta$ is be a function of $x^\mu$.
In the case that $\theta$ is a constant, we deem this a ``global'' phase transformation.
A Lagrangian that is invariant under the transformation in Eqn.~\ref{eqn:local_phase} is said to be \emph{gauge invariant}.
More generally, theories which are invariant under gauge transformations are called \emph{gauge theories}.
As Sec.~\ref{sec:theory_qed} will show, quantum electrodynamics is an abelian gauge theory under the symmetry group U(1), with a single gauge field.
The SM as a whole is a non-abelian gauge theory under the symmetry group U(1) $\times$ SU(2) $\times$ SU(3), with a total of twelve gauge fields corresponding to the spin-1 bosons: the photon, the three massive weak bosons, and the eight gluons.

Gauge theories are particularly attractive from a theoretical standpoint for several reasons.
First, demanding gauge invariance seems reasonable a priori -- the transformation in Eqn.~\ref{eqn:local_phase} is simply a change in the coordinate system we use to define the field $\psi$, and the physics of the universe should be independent of the particular choice of coordinates we use to describe it.
Second, gauge theories have been proven to be renormalizable~\cite{tHooft:1971qjg}, also a reasonable requirement for a theory we hope will describe the universe.

\subsection{Quantum Electrodynamics} \label{sec:theory_qed}
Starting with the Dirac Lagrangian (Eqn.~\ref{eqn:dirac_lagrangian}), suppose we impose that its resulting field equations must be invariant under a local phase transformation (as given by Eqn.~\ref{eqn:local_phase}).
Initially, the Dirac Lagrangian is not invariant under the local phase transformation, as an extra term from the derivative of $\theta$ appears:
\begin{equation} \label{eqn:ld_var}
    \mathcal L \to \mathcal L - q (\partial_\mu \theta)\bar{\psi} \gamma^\mu \psi.
\end{equation}
The situation can be remedied with the introduction of a vector field $A_\mu$ which transforms under Eqn.~\ref{eqn:local_phase} as
\begin{equation} \label{eqn:local_phase_field}
    A_\mu \to A_\mu + \partial_\mu \theta.
\end{equation}
The resulting Lagrangian,
\begin{equation} \label{eqn:ld_inv}
    \mathcal L = \mathcal L_{\text{Dirac}} + q \bar{\psi} \gamma^\mu \psi A_\mu
\end{equation}
is gauge invariant as the second term in Eqn.~\ref{eqn:ld_inv} cancels exactly with the additional term from the transformation to the field $A_\mu$ in Eqn.~\ref{eqn:local_phase_field}.

Frequently this additional field is absorbed into the definition of a \emph{covariant derivative} 
\begin{equation}
    \mathcal D_\mu \equiv \partial_\mu + i q A_\mu
\end{equation}
which replaces the original definition of the derivative, and the resulting field equations are then invariant under local phase transformations, as desired.

The vector field $A_\mu$ which has been added to the Lagrangian implies the existence of an associated spin-1 particle.
In principle, we must also include a free term for the field $A_\mu$: it is natural to start with the Proca Lagrangian (Eqn.~\ref{eqn:proca_lagrangian}) which describes the dynamics of free spin-1 particles.
It can be shown~\cite{Griffiths:2008zz} that the mass term in the Proca Lagrangian is not invariant under Eqn.~\ref{eqn:local_phase_field}: this can be interpreted as a requirement that this new vector field $A_\mu$ must be massless.
The full Lagrangian becomes
\begin{align}
    \mathcal L_{\text{QED}} &= \bar{\psi}(i \gamma_\mu \partial_\mu - m) \psi - \frac{1}{4} F^{\mu\nu}F_{\mu\nu} + q (\bar{\psi}\gamma^\mu \psi)A_\mu, \\
                            &= \bar{\psi}(i \gamma_\mu \mathcal D_\mu - m)\psi - \frac{1}{4} F^{\mu\nu}F_{\mu\nu}
\end{align}
with $F^{\mu\nu} = \partial^\mu A^\nu - partial^\nu A^\mu$, which can be identified as the Lagrangian for quantum electrodynamics.
The field $A_\mu$ is associated with the photon, the constant $q$ with the charge of the electron, the tensor $F^{\mu\nu}$ with the electromagnetic field strength, and interactions between photons and electrons with the trilinear term $q (\bar{\psi}\gamma^\mu \psi)A_\mu$. 

\subsection{Quantum Chromodynamics} \label{sec:theory_qcd}
As Sec.~\ref{sec:pp_parton_model} details, inelastic scattering experiments in the 1960s gave strong evidence of the composite nature of protons.
Zweig~\cite{Zweig:1964jf} and Gell-Mann~\cite{GellMann:1964nj} independently proposed a quark model to describe the composite nature, which initially implied that quarks violate the spin-statistics theorem.
The remedy to this came in the proposal~\cite{Greenberg:1964pe} that each quark comes in three different \emph{colors}.
More formally, this is the statement that quarks are assigned to the fundamental representation $SU(3)$, giving rise to a quantum number which has three states which we arbitrarily call \emph{red}, \emph{green}, and \emph{blue}.

In attempting to construct the Lagrangian for quantum chromodynamics (QCD), which describes the strong interaction of quarks, we can again begin with the free Dirac Lagrangian for spin-1/2 particles (Eqn.~\ref{eqn:dirac_lagrangian}).
Given that we have three distinct colors of each quark, the free Lagrangian for a particular flavor is actually a sum of three free Dirac Lagrangians.
This is simplified with the notation
\begin{equation}
    \psi = \begin{bmatrix}
        \psi_r \\
        \psi_b \\
        \psi_g \\
    \end{bmatrix},
    \quad\bar{\psi} = [\bar{\psi}_r ~\bar{\psi}_b ~\bar{\psi}_g]
\end{equation}
in which the spinor $\psi$ from the original Dirac Lagrangian has now been promoted to a three-component column vector.
The single-particle Dirac Lagrangian is invariant under global phase transformations; in other words, it has $U(1)$ invariance.
Similarly, the three-particle Dirac Lagrangian has $U(3)$ invariance:
\begin{equation} \label{eqn:u3_global}
    \psi \to U \psi, \quad \bar{\psi} \to \bar{\psi}U^\dagger
\end{equation}
with $U$ any unitary $3 \times 3$ matrix\footnote{A matrix $U$ is said to be \emph{unitary} if $U^\dagger U = 1$}.
Whereas in the case of $U(1)$ symmetry, the invariance has the simple interpretation of a phase, the picture is more subtle for $U(3)$.
It can be shown~\cite{Griffiths:2008zz} that any unitary matrix can be written in the form
\begin{equation}
    U  = e^{i \theta} e^{i \bf{\lambda} \cdot \bf{a}}
\end{equation}
with
\begin{equation}
    \bf{\lambda} \cdot \bf{a} = \sum_{i=1}^8 \lambda_i a_i
\end{equation}
and the matrices $a_i$ identified with the eight Gell-Mann matrices which are the generators of the group $SU(3)$.
Following the development of the QED Lagrangian, we again impose the requirement that the Lagrangian not just be invariant under global transformations as described by Eqn.~\ref{eqn:u3_global}, but also local transformations.
In other words, we want $\mathcal L$ to be invariant under local $SU(3)$ gauge transformations:
\begin{equation}
    \psi \to S \psi, \qquad S \equiv e^{-ig\bf{\lambda} \cdot \phi(x)} \text{ and } \phi \equiv - \frac{1}{g_s} \bf{a}
\end{equation}
As in the case of QED, this can be accomplished through the definition of a covariant derivative
\begin{equation}
    \mathcal D_\mu \equiv \partial_\mu + i g_s \bf{\lambda} \cdot \bf{a},
\end{equation}
resulting in the following Lagrangian which is now invariant under local gauge transformations:
\begin{equation}
    \mathcal L = \bar{\psi}(i \gamma_\mu \mathcal D_\mu - m) \psi.
\end{equation}
This time, we have introduced eight gauge fields $\bf{A}^\mu$, corresponding to the eight gluons.

Finally, we must account for the free gluon field.
As before, the mass terms are excluded because they violate local gauge invariance.
However, the field strength tensor for QED, $F^{\mu\nu} = \partial^\mu A^\nu - partial^\nu A^\mu$, cannot be directly generalized to QCD due to the fact that transformations of $SU(3)$ are non-Abelian.
An additional term is required to restore local gauge invariance, resulting in the QCD field strength tensor
\begin{equation}
    F_{\mu \nu}^a = \partial_\mu A_\nu^a - \partial_\nu A_\mu^a - g_s f^{abc} A_\mu^b A_\mu^c.
\end{equation}
The term $f^{abc}$ corresponds to the SU(3) structure constants and are defined by the commutation relation $[\lambda_a, \lambda_b] = i f^{abc} \lambda_c$.
This has no analog in QED; it allows for self-interaction of gluons.
The full QCD Lagrangian is then given by
\begin{equation} \label{eqn:qcd_lagrangian}
    \mathcal L_{\text{QCD}} = \bigg(\sum_f \bar{\psi}_f (i \gamma_\mu \mathcal D_\mu - m_f) \psi_f \bigg) - \frac{1}{4} F_{\mu \nu}^a F^{a \mu \nu},
\end{equation}
where the sum over $f$ corresponds to the different flavors of quarks, of which six have been experimentally observed.

Unlike in QED, in which the magnitude of the force associated with the free photon field \emph{decreases} with distance, the magnitude of the strong force associated with the free gluon field \emph{increases} as a function of distance.
As a result, particles possessing color charge cannot exist as free particles and are instead confined to bound states of multiple particles which must always be colorless.

\subsection{Spontaneous Symmetry Breaking \& The Higgs Mechanism} \label{sec:theory_ssbhm} 
We were able to derive the Lagrangians for the electromagnetic and strong interactions by starting with the Dirac Lagrangian describing free spin-1/2 particles and imposing the principle of local gauge invariance.
This involved the introduction of additional vector fields, with which we are able to associate the mediators of each force: the photon (for QED) and the eight gluons (for QCD).
The fact that the mass term in the Proca Lagrangian is not locally gauge invariant implies that the mediators must be massless; conveniently, photons and gluons are indeed observed to be massless.
Given the success of the method of imposing local gauge invariance for deriving the Lagrangians for the electromagnetic and strong interactions, it is natural to extend the method to the weak interaction.
An immediate challenge, however, is the fact that the mediators of the weak interaction, the $W$ and $Z$ bosons, are not massless.
Local gauge invariance can still be applied to the weak interaction, but it requires reinterpreting the original field variables in a form that allows us to expand about their ground state.
By doing so, we find that symmetries in the original Lagrangian are broken because of the fact that the ground state does not share the symmetry of the original Lagrangian.
This allows for locally invariant massive gauge fields, and as a consequence, implies the presence of a massive scalar particle, which we will identify with the Higgs boson.

\subsection{Electroweak Interactions} \label{sec:theory_ewk}

