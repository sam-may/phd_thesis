%The standard model of particle physics is a quantum field theory which describes three of the four known fundamental forces and all known elementary particles.
%The SM describes the electromagnetic, strong, and weak interactions, but does not provide a description of gravity.
As previously mentioned, the standard model of particle physics is a quantum field theory which describes three of the four known fundamental forces: electromagnetic, weak, and strong.
In particular, the SM is a gauge field theory, meaning its Lagrangian is invariant under certain local transformations.
Gauge fields are discussed in greater detail in Sec.~\ref{sec:theory_gauge}.
In Sec.~\ref{sec:theory_qed}, we will see that imposing local gauge invariance on the Dirac Lagrangian gives rise to quantum electrodynamics.
Sec.~\ref{sec:theory_qcd} describes the strong interaction and Sec.~\ref{sec:theory_ssbhm} illustrates how spontaneous symmetry breaking and the Higgs mechanism allow for massive gauge fields, enabling us to describe the weak interaction in Sec.~\ref{sec:theory_ewk}.
%Sec.~\ref{sec:theory_qcd} describes the strong interaction and Sec.~\ref{sec:theory_ewk} describes the electroweak interaction, the unification of the electromagnetic and weak interactions.
%Finally, in Sec.~\ref{sec:theory_ssbhm} we will see how the Higgs mechanism allows for massive gauge fields and subsequently generates the masses of the gauge bosons and fermions.

\subsection{Gauge Fields} \label{sec:theory_gauge}
For an arbitrary Lagrangian made of a single field variable $\psi$, suppose we impose that its resulting field equations be invariant under the local phase transformation
\begin{equation} \label{eqn:local_phase}
    \psi \to e^{i q \theta(x)} \psi.
\end{equation}
This is deemed a ``local'' phase transformation as the phase $\theta$ is be a function of $x^\mu$.
In the case that $\theta$ is a constant, we deem this a ``global'' phase transformation.
A Lagrangian that is invariant under the transformation in Eqn.~\ref{eqn:local_phase} is said to be \emph{gauge invariant}.
More generally, theories which are invariant under gauge transformations are called \emph{gauge theories}.
As Sec.~\ref{sec:theory_qed} will show, quantum electrodynamics is an abelian gauge theory under the symmetry group U(1), with a single gauge field.
The SM as a whole is a non-abelian gauge theory under the symmetry group U(1) $\times$ SU(2) $\times$ SU(3), with a total of twelve gauge fields corresponding to the spin-1 bosons: the photon, the three massive weak bosons, and the eight gluons.

Gauge theories are particularly attractive from a theoretical standpoint for several reasons.
First, demanding gauge invariance seems reasonable a priori -- the transformation in Eqn.~\ref{eqn:local_phase} is simply a change in the coordinate system we use to define the field $\psi$, and the physics of the universe should be independent of the particular choice of coordinates we use to describe it.
Second, gauge theories have been proven to be renormalizable~\cite{tHooft:1971qjg}, also a reasonable requirement for a theory we hope will describe the universe.

Renormalization refers to the technique by which a quantum field theory is ``cut off'' above some very high energy scale $\Lambda$, above which the theory is assumed to no longer be valid.
In general, this is motivated by the presence of infinities in perturbative calculations of decay rates and cross sections.
Rather than assume these infinities render the Lagrangian a useless description of our universe, renormalization serves as a way of quantitatively applying the qualitative statement that the Lagrangian is a low-energy approximation of a more fundamental theory.
By formalizing the idea that the theory is only valid up to a certain energy scale, we are able to avoid the presence of infinities in the calculation of decay rates and cross sections.

\subsection{Quantum Electrodynamics} \label{sec:theory_qed}
Starting with the Dirac Lagrangian (Eqn.~\ref{eqn:dirac_lagrangian}), suppose we impose that its resulting field equations must be invariant under a local phase transformation (as given by Eqn.~\ref{eqn:local_phase}).
Initially, the Dirac Lagrangian is not invariant under the local phase transformation, as an extra term from the derivative of $\theta$ appears:
\begin{equation} \label{eqn:ld_var}
    \mathcal L \to \mathcal L - q (\partial_\mu \theta)\bar{\psi} \gamma^\mu \psi.
\end{equation}
The situation can be remedied with the introduction of a vector field $A_\mu$ which transforms under Eqn.~\ref{eqn:local_phase} as
\begin{equation} \label{eqn:local_phase_field}
    A_\mu \to A_\mu + \partial_\mu \theta.
\end{equation}
The resulting Lagrangian,
\begin{equation} \label{eqn:ld_inv}
    \mathcal L = \mathcal L_{\text{Dirac}} + q \bar{\psi} \gamma^\mu \psi A_\mu
\end{equation}
is gauge invariant as the second term in Eqn.~\ref{eqn:ld_inv} cancels exactly with the additional term from the transformation to the field $A_\mu$ in Eqn.~\ref{eqn:local_phase_field}.

Frequently this additional field is absorbed into the definition of a \emph{covariant derivative} 
\begin{equation}
    \mathcal D_\mu \equiv \partial_\mu + i q A_\mu
\end{equation}
which replaces the original definition of the derivative, and the resulting field equations are then invariant under local phase transformations, as desired.

The vector field $A_\mu$ which has been added to the Lagrangian implies the existence of an associated spin-1 particle.
In principle, we must also include a free term for the field $A_\mu$: it is natural to start with the Proca Lagrangian (Eqn.~\ref{eqn:proca_lagrangian}) which describes the dynamics of free spin-1 particles.
It can be shown~\cite{Griffiths:2008zz} that the mass term in the Proca Lagrangian is not invariant under Eqn.~\ref{eqn:local_phase_field}: this can be interpreted as a requirement that this new vector field $A_\mu$ must be massless.
The full Lagrangian becomes
\begin{align}
    \mathcal L_{\text{QED}} &= \bar{\psi}(i \gamma_\mu \partial_\mu - m) \psi - \frac{1}{4} F^{\mu\nu}F_{\mu\nu} + q (\bar{\psi}\gamma^\mu \psi)A_\mu, \\
                            &= \bar{\psi}(i \gamma_\mu \mathcal D_\mu - m)\psi - \frac{1}{4} F^{\mu\nu}F_{\mu\nu}
\end{align}
with $F^{\mu\nu} = \partial^\mu A^\nu - \partial^\nu A^\mu$, which can be identified as the Lagrangian for quantum electrodynamics.
The field $A_\mu$ is associated with the photon, the constant $q$ with the charge of the electron, the tensor $F^{\mu\nu}$ with the electromagnetic field strength, and interactions between photons and electrons with the trilinear term $q (\bar{\psi}\gamma^\mu \psi)A_\mu$.

It is instructive to reflect on the implications of demanding gauge invariance: we started with a spinor field characterized by the Dirac Lagrangian, as would be natural to do in attempting to describe the behavior of electrons.
Next, by simply requiring that the equations of motion be invariant under changes in the coordinate system used to describe the field (i.e. demanding gauge invariance), we see that there must be an accompanying massless vector field which interacts with the spinor field, which we identify as the photon field.
This is truly remarkable: we have inferred the existence of photons just by demanding that the behavior of electrons be independent of the choice of coordinate system used to describe their field.

\subsection{Quantum Chromodynamics} \label{sec:theory_qcd}
As Sec.~\ref{sec:pp_parton_model} details, inelastic scattering experiments in the 1960s gave strong evidence of the composite nature of protons.
Zweig~\cite{Zweig:1964jf} and Gell-Mann~\cite{GellMann:1964nj} independently proposed a quark model to describe the composite nature, which initially implied that quarks violate the spin-statistics theorem.
The remedy to this came in the proposal~\cite{Greenberg:1964pe} that each quark comes in three different \emph{colors}.
More formally, this is the statement that quarks are assigned to the fundamental representation $SU(3)$, giving rise to a quantum number which has three states which we (arbitrarily) call \emph{red}, \emph{green}, and \emph{blue}.

In attempting to construct the Lagrangian for quantum chromodynamics (QCD), which describes the strong interaction of quarks, we can again begin with the free Dirac Lagrangian for spin-1/2 particles (Eqn.~\ref{eqn:dirac_lagrangian}).
Given that we have three distinct colors of each quark, the free Lagrangian for a particular flavor is actually a sum of three free Dirac Lagrangians.
This is simplified with the notation
\begin{equation}
    \psi = \begin{bmatrix}
        \psi_r \\
        \psi_b \\
        \psi_g \\
    \end{bmatrix},
    \quad\bar{\psi} = [\bar{\psi}_r ~\bar{\psi}_b ~\bar{\psi}_g]
\end{equation}
in which the spinor $\psi$ from the original Dirac Lagrangian has now been promoted to a three-component column vector.
The single-particle Dirac Lagrangian is invariant under global phase transformations; in other words, it has $U(1)$ invariance.
Similarly, the three-particle Dirac Lagrangian has $U(3)$ invariance:
\begin{equation} \label{eqn:u3_global}
    \psi \to U \psi, \quad \bar{\psi} \to \bar{\psi}U^\dagger
\end{equation}
with $U$ any unitary $3 \times 3$ matrix\footnote{A matrix $U$ is said to be \emph{unitary} if $U^\dagger U = 1$}.
Whereas in the case of $U(1)$ symmetry, the invariance has the simple interpretation of a phase, the picture is more subtle for $U(3)$.
It can be shown~\cite{Griffiths:2008zz} that any unitary matrix can be written in the form
\begin{equation}
    U  = e^{i \theta} e^{i \bf{\lambda} \cdot \bf{a}}
\end{equation}
with
\begin{equation}
    \bf{\lambda} \cdot \bf{a} = \sum_{i=1}^8 \lambda_i a_i
\end{equation}
and the matrices $a_i$ identified with the eight Gell-Mann matrices which are the generators of the group $SU(3)$.
Following the development of the QED Lagrangian, we again impose the requirement that the Lagrangian not just be invariant under global transformations as described by Eqn.~\ref{eqn:u3_global}, but also local transformations.
In other words, we want $\mathcal L$ to be invariant under local $SU(3)$ gauge transformations:
\begin{equation}
    \psi \to S \psi, \qquad S \equiv e^{-ig\bf{\lambda} \cdot \phi(x)} \text{ and } \phi \equiv - \frac{1}{g_s} \bf{a}
\end{equation}
As in the case of QED, this can be accomplished through the definition of a covariant derivative
\begin{equation}
    \mathcal D_\mu \equiv \partial_\mu + i g_s \bf{\lambda} \cdot \bf{a},
\end{equation}
resulting in the following Lagrangian which is now invariant under local gauge transformations:
\begin{equation}
    \mathcal L = \bar{\psi}(i \gamma_\mu \mathcal D_\mu - m) \psi.
\end{equation}
This time, we have introduced eight gauge fields $\bf{A}^\mu$, corresponding to the eight gluons.

Finally, we must account for the free gluon field.
As before, the mass terms are excluded because they violate local gauge invariance.
However, the field strength tensor for QED, $F^{\mu\nu} = \partial^\mu A^\nu - \partial^\nu A^\mu$, cannot be directly generalized to QCD due to the fact that transformations of $SU(3)$ are non-Abelian.
An additional term is required to restore local gauge invariance, resulting in the QCD field strength tensor
\begin{equation}
    F_{\mu \nu}^a = \partial_\mu A_\nu^a - \partial_\nu A_\mu^a - g_s f^{abc} A_\mu^b A_\mu^c.
\end{equation}
The term $f^{abc}$ corresponds to the SU(3) structure constants and are defined by the commutation relation $[\lambda_a, \lambda_b] = i f^{abc} \lambda_c$.
This has no analog in QED; it allows for self-interaction of gluons.
The full QCD Lagrangian is then given by
\begin{equation} \label{eqn:qcd_lagrangian}
    \mathcal L_{\text{QCD}} = \bigg(\sum_f \bar{\psi}_f (i \gamma_\mu \mathcal D_\mu - m_f) \psi_f \bigg) - \frac{1}{4} F_{\mu \nu}^a F^{a \mu \nu},
\end{equation}
where the sum over $f$ corresponds to the different flavors of quarks, of which six have been experimentally observed.

Unlike in QED, in which the magnitude of the force associated with the free photon field \emph{decreases} with distance, the magnitude of the strong force associated with the free gluon field \emph{increases} as a function of distance.
As a result, particles possessing color charge cannot exist as free particles and are instead confined to bound states of multiple particles which must always be colorless.

\subsection{Spontaneous Symmetry Breaking \& The Higgs Mechanism} \label{sec:theory_ssbhm} 
We were able to derive the Lagrangians for the electromagnetic and strong interactions by starting with the Dirac Lagrangian describing free spin-1/2 particles and imposing the principle of local gauge invariance.
This involved the introduction of additional vector fields, with which we are able to associate the mediators of each force: the photon (for QED) and the eight gluons (for QCD).
The fact that the mass term in the Proca Lagrangian is not locally gauge invariant implies that the mediators must be massless; conveniently, photons and gluons are indeed observed to be massless.
Given the success of the method of imposing local gauge invariance for deriving the Lagrangians for the electromagnetic and strong interactions, it is natural to extend the method to the weak interaction.
An immediate challenge, however, is the fact that the mediators of the weak interaction, the $W$ and $Z$ bosons, are not massless.
Local gauge invariance can still be applied to the weak interaction, but it requires reinterpreting the original field variables in a form that allows us to expand about their ground state.
By doing so, we find that symmetries in the original Lagrangian are broken because of the fact that the ground state does not share the symmetry of the original Lagrangian.
This allows for locally invariant massive gauge fields, and as a consequence, implies the presence of a massive scalar particle, which we will identify with the Higgs boson.

Spontaneous symmetry breaking and the Higgs mechanism can be illustrated through a toy Lagrangian composed of a single complex field:
\begin{equation} \label{eqn:spon_orig}
    \mathcal L = \frac{1}{2} (\partial_\mu \phi)^*(\partial^\mu \phi) + \frac{1}{2} \mu^2 (\phi^* \phi) - \frac{1}{4} \lambda^2 (\phi^* \phi)^2, \qquad \phi \equiv \phi_1 + i \phi_2
\end{equation}
In this Lagrangian, the mass term $(1/2) \mu^2 (\phi^* \phi)$ appears to have the wrong sign: naively, a positive coefficient implies that the particle associated with the $\phi$ field has an imaginary mass.
Physically, this does not make sense.
The subtlety lies in the fact that the Feynman calculus is a perturbative procedure, and must be performed by expanding about a system's ground state.
Interpreting $\frac{1}{2} \mu^2 (\phi^* \phi) - \frac{1}{4} \lambda^2 (\phi^* \phi)^2$ as the \emph{potential} term in the Lagrangian, we can expand about its minimum and apply the Feynman calculus.
In contrast to previously considered fields, the minimum does not occur at $\phi_1 = \phi_2 = 0$, but rather is defined by the circle
\begin{equation} \label{eqn:spon_gs}
    \phi_{1_{\text{min}}}^2 + \phi_{2_{\text{min}}}^2 = \frac{\mu^2}{\lambda^2}.
\end{equation}
Choosing $\phi_{1_{\text{min}}} = \mu/\lambda$ and $\phi_{2_{\text{min}}} = 0$, let us next rewrite the Lagrangian in terms of fields which can be treated as fluctuations about the vacuum state, defining
\begin{equation} \label{eqn:spon_gs}
    \eta \equiv \phi_1 - \frac{\mu}{\lambda}, \qquad \xi \equiv \phi_2.
\end{equation}
In terms of these new fields, the Lagrangian becomes
\begin{equation} \label{eqn:spon_transf}
    \mathcal L = \frac{1}{2} (\partial_\mu \eta)(\partial^\mu \eta) - \mu^2 \eta^2 + \frac{1}{2} (\partial_\mu \xi)(\partial^\mu \xi) - \mu \lambda(\eta^3 + \eta \xi^2) - \frac{\lambda^2}{4}(\eta^4 + \xi^4 + 2 \eta^2 \xi^2) + \frac{\mu^4}{4 \lambda^2}.
\end{equation}
The original Lagrangian (Eqn.~\ref{eqn:spon_orig}) was invariant under rotations in $\phi_1, \phi_2$ space\footnote{More precisely, the original Lagrangian is invariant under SO(2)}; however, this rotational symmetry is no longer manifest in the $\eta, \xi$ space.
The continuous SO(2) symmetry has been broken by the choice of a particular ground state.
The particular ground state we chose, $\phi_{1_{\text{min}}} = \mu/\lambda$ and $\phi_{2_{\text{min}}} = 0$, is arbitrary: the system could just as easily choose any other ground state which satisfies Eqn.~\ref{eqn:spon_gs}.
For this reason, we say that the symmetry has been \emph{spontaneously} broken.

Examining Eqn.~\ref{eqn:spon_transf}, we can identify that the particle associated with the $\eta$ field has mass $m_\eta = \sqrt{2} \mu$ and that the particle associated with the $\xi$ field is massless.
In fact, Goldstone's theorem~\cite{Goldstone:1962es} shows that the spontaneous breaking of a continuous global symmetry is associated with one or more massless scalar particles, referred to as \emph{Goldstone bosons}.

Next, let us impose the condition of local gauge invariance on the original Lagrangian, Eqn.~\ref{eqn:spon_orig}, demanding that it be invariant under transformations of the form $\phi \to e^{i \theta(x)} \phi$.
As before, we can introduce a massless gauge field $A^\mu$ and replace derivatives with covariant derivatives to satisfy local gauge invariance.
The Lagrangian becomes
\begin{multline} \label{eqn:spon_full}
    \mathcal L = \frac{1}{2} (\partial_\mu \eta)(\partial^\mu \eta) - \mu^2 \eta^2 + \frac{1}{2} (\partial_\mu \xi)(\partial^\mu \xi) 
    - \frac{1}{4} F^{\mu\nu} F_{\mu\nu} + \frac{1}{2} \bigg(\frac{q\mu}{\lambda}\bigg)^2 A_\mu A^\mu \\
    + q \Big[ \eta(\partial_\mu \xi) - \xi (\partial_\mu \eta) \Big] A^\mu + q^2 \frac{\mu}{\lambda} \eta (A_\mu A^\mu)
    + \frac{1}{2} q^2 (\xi^2 + \eta^2)(A_\mu A^\mu) \\
    - \mu \lambda(\eta^3 + \eta \xi^2) - \frac{\lambda^2}{4}(\eta^4 + \xi^4 + 2 \eta^2 \xi^2) 
    + \frac{\mu}{\lambda}q (\partial_\mu \xi) A^\mu
    + \frac{\mu^4}{4 \lambda^2}.
\end{multline}
The gauge field $A^\mu$ that we introduced to impose local gauge invariance now has a quadratic term $(1/2) (q\mu)\lambda)^2 A_\mu A^\mu$, which we can associate with a \emph{massive} gauge boson.
A mass term associated with the gauge field $A^\mu$ has appeared because of the fact that we have rewritten the Lagrangian in a form that allows us to expand about its ground state.
In terms of the original $\phi_1$ and $\phi_2$ fields, no mass term for $A^\mu$ appears, but once a ground state has been selected (transforming to $\eta$ and $\xi$ fields) the gauge boson associated with $A^\mu$ acquires mass: spontaneous symmetry breaking generates masses for gauge bosons.

The Lagrangian of Eqn.~\ref{eqn:spon_full} still presents some difficulties in its physical interpretation.
There is a bilinear term proportional to $(\partial_\mu \xi)A^\mu$ which we would interpret as allowing for a $\xi$ particle to suddenly become an $A^\mu$ gauge boson.
This implies that we have not yet fully cast the Lagrangian in a form that makes its physical interpretation apparent and can be solved by choosing a particular gauge.
The Lagrangian of Eqn.~\ref{eqn:spon_orig} is invariant under global U(1) phase transformations $\phi \to e^{i\theta} \phi$.
If we choose $\theta = - \tan^{-1}(\phi_2/\phi_1)$, the transformed field $\phi'$ is real ($\phi'_2 = 0$), implying that $\xi = 0$: the problematic bilinear term has been eliminated by the choice of gauge.

We have shown that a gauge boson can acquire mass through the spontaneous breaking of a continuous global symmetry (the SO(2) symmetry of the complex scalar field $\phi$).
With a proper choice of gauge, we identify a real scalar field $\eta$ and a massive scalar particle associated with this field.
This process by which gauge bosons can acquire mass is known as the \emph{Higgs mechanism}~\cite{Higgs:1964pj,Englert:1964et} and the massive scalar is known as a Higgs boson.

\subsection{Electroweak Interactions} \label{sec:theory_ewk}
The weak and electromagnetic interactions can be unified in a single electroweak interaction, originally developed by Glashow, Weinberg, and Salam~\cite{Glashow:1959wxa,Weinberg:1967tq,Salam:1968rm}.
The GWS theory of weak interactions begins with an $SU(2) \otimes U(1)$ gauge symmetry.
The symmetry is broken spontaneously through the introduction of a scalar field, leading to the generation of masses for the gauge bosons of the $SU(2)$ component and leaving the gauge boson of the $U(1)$ symmetry massless.
The former will be identified as the massive vector gauge bosons, the $W^\pm$ and the $Z$, while the latter be identified as the massless photon.
The particle associated with the scalar field responsible for the spontaneous symmetry breaking will be identified as the Higgs boson.

As before, we demand that the Lagrangian be invariant under local gauge transformations, this time of the form $\phi \to e^{i \alpha^a \tau^a} e^{i \beta/2} \phi$ and define a covariant derivative for $\phi$:
\begin{equation}
    D_\mu \phi = (\partial_\mu - i g A^a_\mu \tau^a - \frac{i}{2} g' B_\mu) \phi, \qquad \tau^a = \sigma^a/2
\end{equation}
where $\sigma^a$ are the Pauli spin matrices, $A^a_\mu$ corresponds to the $SU(2)$ gauge bosons and $B_\mu$ corresponds to the $U(1)$ gauge boson.
With a quartic potential for the scalar field $\phi$, as in the example of Sec.~\ref{sec:theory_ssbhm}, the field has a minimum defined by a circle in the $\phi_1, \phi_2$ plane (with $\phi = \phi_1 + i \phi_2$).
The original $SO(2)$ symmetry of $\phi$ will be spontaneously broken when a particular ground state along this circle is chosen.
Assuming a ground state of $\phi_1 = (1/\sqrt{2})v, \phi_2 = 0$, choosing the gauge $\alpha^1 = \alpha^2 = 0,~\alpha^3 = \beta$, and rewriting the Lagrangian about this field configuration leads to the generation of masses for the bosons of the $A^a_\mu$ field and leaves the boson of the $B_\mu$ field massless.
Expressing the Lagrangian in terms of the ground state, rather than the original field $\phi$, we find that the following terms appear:
\begin{equation}
    \Delta \mathcal L = \frac{1}{2} \frac{v^2}{4} \bigg[ g^2 (A^1_\mu)^2 + g^2 (A^2_\mu)^2 + (g' B_\mu - g A^3_\mu)^2 \bigg].
\end{equation}
Again we see that rewriting the Lagrangian in a way such that it can be expanded about its spontaneously chosen ground state breaks the original symmetry and gives rise to mass terms for the $SU(2)$ gauge field. 

The original fields can be expressed in terms of their mass eigenstates, which will make the physical interpretation of this theory more transparent.
It can be shown~\cite{Peskin:1995ev} that these eigenstates are
\begin{align}
    W^\pm_\mu &= \frac{1}{\sqrt{2}} (A^1_\mu \mp i A^2_\mu) \\
    Z^0_\mu &= \frac{1}{\sqrt{g^2 + g'^2}}(g A^3_\mu - g' B_\mu) \\
    A_\mu &= \frac{1}{\sqrt{g^2 + g'^2}}(g'A_\mu^3 + g B_\mu)
\end{align}

The $W^\pm_\mu$ field has vector bosons of mass $m_W = gv/2$, the $Z^0_\mu$ field has a vector boson of mass $m_Z = \sqrt{g^2 + g'^2} v/2$, and the $A_\mu$ field remains massless.
The covariant derivative can be rewritten in terms of the mass eigenstates and the \emph{weak mixing angle}, $\theta_w$, defined as
\begin{equation}
    \begin{pmatrix} Z^0 \\ A \end{pmatrix} =
    \begin{pmatrix} \cos \theta_w & -\sin \theta_w \\ \sin \theta_w & \cos \theta_w \end{pmatrix} \begin{pmatrix} A^3 \\ B \end{pmatrix}.
\end{equation}
As the field $A_\mu$ will be identified as the electromagnetic vector potential, it is helpful to define $e$, which will be identified as the electron charge
\begin{equation}
    e = \frac{gg'}{\sqrt{g^2 + g'^2}}.
\end{equation}
In this notation the covariant derivative becomes
\begin{equation}
    D_\mu \phi = \bigg[ \partial_\mu - i \frac{g}{\sqrt{2}} (W^+_\mu T^+ + W_\mu^- T^-) - i \frac{g}{\cos \theta_w} Z_\mu (T^3 - \sin^2 \theta_w Q) - ieA_\mu Q \bigg] \phi.
\end{equation}

We can next couple the $SU(2) \otimes U(1)$ gauge fields of the electroweak interaction to the leptons and quarks.
As the W boson couples only to the left-handed helicity states of leptons and quarks, it is helpful to decompose the kinetic energy term for fermions into left- and right-handed components:
\begin{equation}
    \bar{\psi} i \gamma^\mu \partial_\mu \psi = \bar{\psi}_L i \gamma^\mu \partial_\mu \psi_L + \bar{\psi}_R i \gamma^\mu \partial_\mu \psi_R,
\end{equation}
so that $\psi_L$ and $\psi_R$ can couple differently to the gauge fields.

The left- and right-handed components of fermion fields couple differently to the gauge fields, they have different quantum numbers and consequently simple mass terms are forbidden by gauge invariance.
Experimentally, we know that the fermions are not massless, so this poses a problem.
Again, spontaneous symmetry breaking can remedy this and allow for fermions to acquire mass.

Assuming the scalar field $\phi$ undergoes spontaneous symmetry breaking (``acquires a vacuum expectation value''), we can add terms to the Lagrangian which describe interactions between $\phi$ and left- and right-handed components of fermions.
For example, for the electrons:
\begin{equation}
\Delta \mathcal L_e = - \lambda_e \bar{E}_L \cdot \phi e_R + \text{ h.c.}, \qquad E_L = \begin{pmatrix} \nu_e \\ e^- \end{pmatrix} 
\end{equation}
Assuming the same ground state as before, $\phi = (0 \quad v/\sqrt{2})$, we obtain a mass term for the electron:
\begin{equation}
    \Delta \mathcal L_e = \frac{-1}{\sqrt{2}} \lambda_e v \bar{e}_L e_R + \text{ h.c.},
\end{equation}
referred to as the Yukawa term and $\lambda_e$ referred to as the Yukawa coupling.
Yukawa terms for the other leptons and the quarks can be obtained in a similar fashion, such that the mass of any fermion is given by
\begin{equation}
    m_f = \frac{1}{\sqrt{2}} \lambda_f v.
\end{equation}
Neither the magnitude of the Yukawa couplings, $\lambda_f$, nor the vacuum expectation value, $v$, are known a priori and must be measured experimentally.

One final subtlety of the electroweak interaction is that while the W boson couples to leptons only of the same generation, it can couple to quarks from different generations.
This mixing is a result of the fact that the mass eigenstates of quarks are different from the weak isospin eigenstates.
The mixing of these eigenstates is described by the Cabibbo-Kobayashi-Maskawa (CKM) matrix~\cite{Cabibbo:1963yz,Kobayashi:1973fv}:
\begin{equation}
V_{\text{CKM}} = \begin{pmatrix} V_{ud} & V_{us} & V_{ub} \\ V_{cd} & V_{cs} & V_{cb} \\ V_{td} & V_{ts} & V_{tb} \end{pmatrix}.
\end{equation}
The CKM matrix is experimentally observed to be nearly diagonal, which has the physical consequence that W-mediated interactions between quarks of different generations are much weaker than those between quarks of the same generation.

\subsection{Shortcomings of the Standard Model} \label{sec:theory_sm_problems}
Though the SM has proved to be compatible with nature in nearly every single experimental test, it is known to be an incomplete theory.
Some of the most glaring shortcomings of the SM are summarized in this section, but it is by no means an exhaustive list of every problem.

First, the SM does not provide a description of gravity, and is (so far) incompatible with the theory of \emph{general relativity}, currently the most successful theory of gravity.
Second, the SM cannot explain dark matter \& dark energy, with strong evidence for dark matter given by the inconsistency of galactic rotation curves with the amount of visible matter~\cite{Rubin:1980zd}.
Third, the SM does not explain the observed dominance of matter over antimatter in the universe~\cite{Dine:2003ax} -- CP violation (which is allowed and observed in the SM) can account for some of this asymmetry, but not nearly enough to explain the observed asymmetry.
Finally, the SM is known to be self-inconsistent at very high energies, with the electromagnetic coupling and the Higgs self-coupling both diverging at arbitrarily high energies~\cite{Gockeler:1997dn,Chivukula:1999az}.
This issue, called \emph{quantum triviality}, implies that the SM is only valid up to some finite energy scale.

Given the abundance of issues with the SM, it is widely believed that the SM is a low-energy approximation of some more fundamental theory.
With the stipulation that the SM is fundamentally an incorrect description of the universe, it is natural to search for the ways in which the SM does \emph{not} correctly describe our universe.
In this paradigm, the experimental success of the SM is puzzling -- if the SM is wrong, why does every experimental test agree with its predictions?
One of the more promising approaches is to test the SM at increasingly high energy scales, as is done through analysis of the 13 TeV center-of-mass collisions at the Large Hadron Collider (LHC).
Among the countless measurements which can be made with the datasets from the LHC's numerous experiments, measuring the properties of the Higgs boson is a promising avenue to search for departures from the SM predictions given the Higgs unique role in mass generation and electroweak symmetry breaking.
