This section describes the standard model of particle physics, currently the best known description of the universe's fundamental particles and their interactions.
Sec.~\ref{sec:theory_qft} describes quantum field theory, the theoretical framework upon which the standard model (SM) is founded.
Details of the SM are then described in Sec.~\ref{sec:theory_sm}, with a focus on the central role played by spontaneous symmetry breaking and the Higgs mechanism.
Finally, shortcomings of the SM are discussed in Sec.~\ref{sec:theory_sm_problems}, motivating the Higgs boson as a tool to search for new physics beyond the SM.

The SM is a quantum field theory which describes three of the four known fundamental forces and all known elementary particles.
It describes the electromagnetic, strong, and weak interactions, but does not provide a description of gravity.
The SM particles can be initially categorized into two groups, bosons and fermions, defined by their intrinsic angular momentum, called ``spin''.

Bosons are particles which have integer quantum numbers for spin, while fermions are particles which have half-integer quantum numbers for spin.
Except for the Higgs boson, a spin-0 ``scalar'', all bosons in the SM have spin-1.
Each of the three forces described by the SM are mediated by the spin-1 gauge bosons: the photon for the electromagnetic force, the $W^\pm$ and $Z$ bosons for the weak force, and the eight gluons for the strong force.
%Table~\ref{tab:bosons} summarizes the properties of the SM bosons.

The SM fermions all have spin-$1/2$ and can be further divided into two categories: leptons and quarks.
Quarks participate in the strong interaction, while leptons do not.
Quarks also participate in the electromagnetic and weak interactions.
There are both ``up''-type (positively charged) and ``down''-type (negatively charged) quarks, with three generations of each, giving six distinct quarks.
Each quark also comes in three ``color'' varieties; however, the different colors of quarks are not experimentally distinct from one another.
Leptons can also be further divided into two categories: those which interact with the electromagnetic force (electrons, muons, and taus) and those which interact only with the weak force (neutrinos).
Furthermore, each particle in the SM has an accompanying \emph{antiparticle} with opposite electric charge and parity, but otherwise identical physical properties.
Table~\ref{tab:sm_particles} summarizes the properties of the SM particles.

\begin{table} [htbp!]
    \centering
    \caption{Particle content of the SM, including names, symbol, spin, mass, and electric charge of each particle. Mass values taken from~\cite{Zyla:2020zbs}.}
    \begin{tabular}{ r c r r r r} \hline \hline
        Particle & Symbol & Spin & Mass [GeV] & Electric Charge & Interactions\\ \hline 
        Higgs boson & H & 0 & 125 & 0 & \\ \hline
        Z boson & Z & 1 & 91.2 & 0 & Weak \\ 
        W boson & W & 1 & 80.4 & $\pm$ 1 & Weak\\
        Photon & $\gamma$ & 1 & 0 & 0 & Electromagnetic\\
        Gluon & g & 1 & 0 & 0 & Strong \\ \hline
        Up quark & $u$ & 1/2 & 2.16 $\times 10^{-3}$ & 2/3 & Weak, Electromagnetic, Strong \\
        Charm quark & $c$ & 1/2 & 1.27 & 2/3 & Weak, Electromagnetic, Strong \\
        Top quark & $t$ & 1/2 & 173 & 2/3 & Weak, Electromagnetic, Strong \\
        Down quark & $d$ & 1/2 & 4.67 $\times 10^{-3}$ & -1/3 & Weak, Electromagnetic, Strong \\
        Strange quark & $s$ & 1/2 & 0.093 & -1/3 & Weak, Electromagnetic, Strong \\
        Bottom quark & $b$ & 1/2 & 4.18 & -1/3 & Weak, Electromagnetic, Strong \\ \hline
        Electron & $e$ & 1/2 & 5.11 $\times 10^{-4}$  & -1 & Weak, Electromagnetic \\
        Muon & $\mu$ & 1/2 & 0.106 & -1 & Weak, Electromagnetic\\
        Tau & $\tau$ & 1/2 & 1.78 & -1 & Weak, Electromagnetic\\
        Electron neutrino & $\nu_e$ & 1/2 & < 1.1 $\times 10^{-9}$ & 0 & Weak \\
        Muon neutrino & $\nu_\mu$ & 1/2 & < 1.1 $\times 10^{-9}$ & 0 & Weak \\
        Tau neutrino & $\nu_\tau$ & 1/2 & < 1.1 $\times 10^{-9}$ & 0 & Weak\\ \hline \hline
    \end{tabular}
    %\caption{Particle content of the SM, including names, symbol, spin, mass, and electric charge of each particle. Mass values taken from~\cite{Zyla:2020zbs}.}
    \label{tab:sm_particles}
\end{table}

