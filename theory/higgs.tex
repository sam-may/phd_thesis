The SM hinges upon the existence of a scalar field $\phi$ which undergoes spontaneous symmetry breaking to acquire a vacuum expectation value, thereby allowing the gauge bosons and fermions to acquire mass.
The scalar field must have a potential with minima that lie outside $\phi = 0$ in order for this to occur.
As illustrated in Sec.~\ref{sec:theory_ssbhm}, one such Lagrangian which leads to a vacuum expectation value is
\begin{equation}
    \mathcal L = |D_\mu \phi|^2 + \mu^2 \phi^\dagger \phi - \lambda (\phi^\dagger \phi)^2.
\end{equation}
The particle associated with this field will then have a mass given by
\begin{equation}
    m_H = \sqrt{2} \mu = \sqrt{2 \lambda} v.
\end{equation}
A particle consistent with the Higgs boson was discovered in 2012 by the CMS and ATLAS collaborations~\cite{Aad:2012tfa,Chatrchyan:2012xdj,Chatrchyan:2013lba}, with its mass measured to be $125.35 \pm 0.15$ GeV~\cite{CMS:2019drq}.

Since its discovery, other measurements of the Higgs boson's properties have so far confirmed that it is consistent with the SM Higgs boson.
Multiple production modes of the Higgs boson have been experimentally confirmed at the LHC, with observations of Higgs boson production via gluon fusion and vector boson fusion~\cite{Aad:2013wqa,Khachatryan:2014jba,Khachatryan:2016vau} made during Run 1 of the LHC, and observations of Higgs boson production in association with a vector boson~\cite{Aaboud:2018zhk} or a top quark-antiquark pair~\cite{Sirunyan:2018hoz} made during Run 2 of the LHC.
A variety of expected decay modes of the Higgs boson have also been experimentally confirmed, with branching fractions consistent with the SM predictions.
The $\gamma \gamma$, $ZZ^*$, $W^\pm W^{\mp*}$, $\tau^\pm \tau^\mp$~\cite{Aad:2013wqa,Khachatryan:2014jba,Khachatryan:2016vau}, and $\text{b}\bar{\text{b}}$~\cite{Aaboud:2018zhk} decay modes have each been observed. 

\subsubsection{The Higgs Boson as a probe of new physics}
While the discovery of the Higgs boson and the confirmation of its SM-like properties are some of the most successful experimental validations of the SM, they are also frustrating in some sense:
the SM is known to be an incomplete description of our universe, for reasons detailed in the following section.
It is widely accepted that the SM is a low-energy approximation of some more fundamental theory of the universe.
There are a variety of theories of physics beyond the standard model (BSM), for example, the theory of supersymmetry~\cite{Martin:1997ns}.
Direct searches for new physics have been continually performed at the LHC, but so far, there has been no conclusive evidence for the presence of any BSM physics.
How can we reconcile these two facts: (1) there should be BSM physics and (2) there is no evidence of BSM physics at the LHC?
One possibility is that new physics exists at masses which are beyond the energy reach of the LHC.
In these scenarios, the presence of new physics might still manifest itself in the form of small deviations of the properties of the Higgs boson from the SM expectations.
For example, a two-Higgs doublet model would result in modified coupling constants of the Higgs boson which would be observable at a future particle collider~\cite{Kanemura:2004mg}.

The top quark Yukawa coupling is of particular interest from a theoretical standpoint, as precise measurements of its value could give insight on the existence and energy scale of new physics~\cite{why_care_top_yukawa}.
Measurement of Higgs boson production in association with a top quark-antiquark pair is the best way to directly constrain the top quark Yukawa coupling at the LHC.
One such measurement~\cite{tth_observation} is the focus of this dissertation.

%In these scenarios, precise determination of the properties of the Higgs boson 
%The hypotheses of a spin-2 Higgs boson or a pseudoscalar (spin-0, negative parity) Higgs boson have both been excluded~\cite 
%A particle is associated with this field, the Higgs boson,  
