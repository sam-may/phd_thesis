The field of particle physics saw the discovery of a variety of new particles in the 1950s and 60s.
At the time, their fundamental nature was unknown; however, due to their sheer number it seemed plausible that these particles, now known as mesons and baryons, were not elementary but composite.
Zweig~\cite{Zweig:1964jf} and Gell-Mann~\cite{GellMann:1964nj} independently proposed that mesons and baryons were in fact composed of spin 1/2 particles which Gell-Mann coined ``quarks''.
In this framework, mesons were bound states of a quark and an anti-quark while baryons were bound states of three quarks. More precisely, mesons and baryons are composed of their respective quarks, called valence quarks (which dictate the nucleon's quantum numbers), gluons (which mediate the strong force and bind the nucleon), and a sea of virtual quark-antiquark pairs~\cite{Yan:2015zoa}. 
The quark model of Zweig and Gell-Mann was initially met with some skepticism: it implied that quarks must have fractional charges of either 1/3 or 2/3 the charge of the electron and that they violate the spin-statistics theorem.
The quantum number color~\cite{Greenberg:1964pe} was proposed to remedy the violation of the spin-statistics theorem and deep inelastic scattering experiments at SLAC~\cite{Breidenbach:1969kd,Bloom:1969kc} gave strong indications of the composite structure of the proton.

