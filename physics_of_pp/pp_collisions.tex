\subsection{Cross Sections}
Some of the primary quantities we are interested in predicting and measuring in proton-proton collisions are \emph{cross sections}, a measure of the probability\footnote{More precisely, a cross section is measured in units of distance squared. However, the characteristic cross section of a process is synonymous with its probability of occuring and it is more intuitive to describe cross sections in these terms.} of a specific process occurring.
The QCD factorization theorem~\cite{Collins:1989gx} allows the cross section for an arbitrary deep inelastic proton-proton collision to be written in terms of two components: a perturbatively calculable hard term and a non-perturbative PDF.
Thus, the cross section for $pp \to X + Y$ can be calculated in the following way:
\begin{equation} 
    \sigma(pp \to X + Y) = \sum_{i,j} \int dx_i dx_j f(x_i, Q^2) f(x_j, Q^2) \sigma(q_i q_j \to Y),
\end{equation}
in which $X$ may be any hadronic final state, and $Y$ is an arbitrary final state for the inelastic scattering of two partons $q_i$ and $q_j$.
The sum is calculated over all partons and integrated over all possible momentum fractions for the PDFs of each parton.
The hard term, $\sigma(q_i q_j \to Y)$, can be calculated perturbatively in QCD.
In practice, these calculations are done through the use of Monte Carlo generators, described in greater detail in Sec.~\ref{sec:pp_mc}.
Cross sections for typical processes of interest at pp collision experiments are shown as a function of the center-of-mass energy in Fig.~\ref{fig:pp_xs}.
\begin{figure}[htbp!]
    \centering
    \includegraphics[width=0.6\linewidth]{figures/physics_of_pp/pp_cross_sections.png}
    \caption{Cross sections for typical processes of interest in pp collision experiments, shown as a function of the center-of-mass energy, $\sqrt{s}$. Taken from~\cite{Campbell:2006wx}.}
    \label{fig:pp_xs}
\end{figure}

\subsection{Parton Showers, Hadronization, and Jets} \label{sec:pp_physics_jets}
High energy processes involving the strong interaction are very well-described by perturbative QCD calculations.
However, at lower energies (less than or equal to about 1 GeV), the perturbative appraoch fails to provide an accurate description of the SM phenomena: the strong coupling $\alpha_s$ of QCD becomes close to unity, as shown in Fig.~\ref{fig:pp_qcd_coupling}.
\begin{figure} [htbp!]
    \centering
    \includegraphics[width=0.6\linewidth]{figures/physics_of_pp/pp_qcd_coupling.png}
    \caption{The strong coupling constant $\alpha_s$ of QCD as a function of $Q^2$. Different colored lines correspond to various renormalization schemes. Taken from~\cite{Deur:2016tte}.}
    \label{fig:pp_qcd_coupling}
\end{figure}
When the coupling $\alpha_s$ nears unity, the perturbative approach fails for the following reason: perturbative expansions are made in powers of the coupling, so the coupling must be significantly less than one in order for a finite expansion to provide a good approximation.
In describing phenomena like parton showers and hadronization, energy scales of $\mathcal O(1)$ GeV are relevant, and a strictly perturbative calculation will not provide a satisfactory description.

A \emph{parton shower} refers to the process by which a high energy parton stemming from the hard interaction produce showers of ``soft'' particles at lower energies.
Typically, this is either gluon splitting, in which a gluon converts into a quark-antiquark pair, or gluon radiation, in which a quark radiates a gluon.
In practice, parton showers are modeled with Monte Carlo generators which utilize Sudakov form factors~\cite{Sudakov:1954sw} and splitting functions to simplify calculations~\cite{Hoche:2014rga}.

As discussed in Sec.~\ref{sec:theory_qcd}, quarks and gluons are confined to bound states which must be colorless.
Moreover, the potential energy of a hadron increases as a function of the distance between the partons.
At a large enough distance, it becomes energetically favorable to break the original bound state in which they existed and instead form new hadrons.
This process is called \emph{hadronization}.
In high energy collisions, quarks and gluons are often ejected from the hard interaction with high enough momentum for hadronization to occur. 
Frequently, the newly formed hadron will initiate a cascade of decays and gluon radiation, forming a cone of hadronic activity.
This cone of particles stemming from the hadronization of a quark or gluon is called a \emph{hadronic jet}.
Hadronization cannot be adequately described through perturbative calculations alone, and instead phenomenological models like the Lund-String Model~\cite{Andersson:1983ia} are employed.

\subsection{Underlying Event and Pileup}
In a given bunch crossing, there is typically only one hard scattering interaction of interest from a physics point of view.
In addition to this hard interaction, there are additional lower energy ``soft'' scattering interactions.
The soft scattering may be due either to interactions between partons other than those involved in the hard scattering interaction or interactions between protons other than those involved in the hard scattering interaction.
The former is called the \emph{underlying event}, while the latter interactions are called \emph{pileup} interactions.
The modeling of underlying event and pileup is often performed through heuristic approaches which extrapolate directly from experimental collision data.

Though soft scattering interactions from underlying event and pileup are typically not of interest, it is still imperative to understand and adequately model them in order to study physics processes of interest.
A large portion of the hadronic activity in an event at the LHC stems from these soft interactions and will effect, for example, the jet multiplicity and missing transverse momentum calculation in that event.
Physics analyses often use the jet multiplicity and missing transverse momentum to identify regions of high signal purity (for example, an analysis searching for supersymmetric particles will typically select for events with high missing transverse momentum) -- for these reasons, it is vital to understand the contribution of underlying event and pileup to these distributions in order to properly model the targeted signal process and accurately estimate the relevant SM background processes.

Parton showers, hadronization, and underlying event are visually depicted for a hadron-hadron collision in Fig.~\ref{fig:pp_event_schematic}.

\begin{figure} [htbp!]
    \centering
    \includegraphics[width=0.7\linewidth]{figures/physics_of_pp/pp_event_schematic.png}
    \caption[Schematic of a hadron-hadron collision. Taken from~\cite{Hoche:2014rga}.]{Schematic of a hadron-hadron collision. The red blob indicates the hard scattering interaction and the subsequent tree-like structure depicts parton showers, while the purple blob indicates an underlying event scattering interaction. Light green blobs depict hadronization, dark green blobs depict subsequent decays of those hadrons, and yellow lines depict soft Bremsstrahlung radiation. Taken from~\cite{Hoche:2014rga}.}
    \label{fig:pp_event_schematic}
\end{figure}
